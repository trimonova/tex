В настоящем разделе описывается лабораторная установка, на экспериментальных данных которой будет верифицироваться численная модель. Данная установка была разработана в Институте динамики геосфер Российской академии наук~\cite{trimonova2017, trimonova2018}. К ее основным возможностям можно отнести: исследования гидроразрыва пласта и сопуствующих ему процессов, создание трехмерного напряженно-деформированного состояния, изучение неустойчивых трещин на нагнетательных скважинах, моделирование взаимодействия трещин и многое другое. Ниже будут описаны особенности уставновки, важные с точки зрения математического моделирования протекающих в ней процессов.

\subsection{Описание установки}
Конструктивно установка для моделирования процессов в плоском коллекторе состоит из верхней и нижней крышек, в кольцевом углублении которых фиксируется боковина.      Схематический внешний вид установки показан на рисунке~\ref{device:pict}. Между собой крышки скрепляются шестнадцатью шпильками, которые на рисунке не показаны. Толщина крышек составляет  75~мм при наружном диаметре 600~мм. Высота боковины равна 70~мм при внутреннем диметре 430~мм и толщине стенки 25~мм.

\begin{figure}[hb]
\begin{center}
\epsfig{file=figs/picts/1, width=.6\textwidth}
\end{center}
\caption{Схема плоской модельной установки}\label{device:pict}
\end{figure}

Основные детали установки изготовлены из нержавеющей стали. Крышки размещаются на станине, что позволяет свободно их переворачивать и перемещать верхнюю крышку в свой ложемент. Это дает возможность проводить необходимые операции, связанные с подготовкой и проведением экспериментов, несмотря на значительную массу составных частей установки (масса крышек превышает 160 кг). Фотографии общего вида установки представлены на рисунках~\ref{device1:pict}.

\begin{figure}[hb]
\begin{center}
\begin{tabular}{cc}
\epsfig{file=figs/picts/2, height=6cm} &
\epsfig{file=figs/picts/3, height=6cm} 
\end{tabular}
\end{center}
\caption{Общий вид экспериментальной установки}\label{device1:pict}
\end{figure}

Перед началом эксперимента в установку заливается гипс. Затвердевая, он образует модельную среду. Верхняя крышка отделена от образца резиновой мембраной. По периметру мембраны расположены резиновое кольцо и опорный хомут, которые создают герметичное пространство между мембраной и крышкой (рис. ~\ref{device:pict}). Это пространство заполняется водой под давлением, что позволяет моделировать литостатическое давление в модели коллектора. Давление над мембраной поддерживается при помощи разделительного цилиндра, верхняя часть которого заполнена сжатым азотом под необходимым давлением, а нижняя водой. 

Горизонтальное нагружение модели обеспечивается с помощью герметичных камер, расположенных на поверхности боковой стенки. Фотография установки (вид сверху) с боковыми камерами представлен на рисунке ~\ref{device2:pict}. Камеры изготовлены из листовой меди толщиной 0,3~мм. Внутренняя полость камер имеет толщину 3~мм, высота камеры на 2~мм меньше высоты боковой стенки. Длина дуги камеры составляет примерно $80^\circ$. Патрубок камеры через герметичное уплотнение выводится из боковой стенки наружу. Камеры зафиксированы на боковой поверхности кольца с помощью силиконового герметика. Боковое нагружение осуществляется за счет закачки газа или жидкости в попарно противоположные камеры.

\begin{figure}[hb]
\begin{center}
\epsfig{file=figs/picts/4, width=.4\textwidth}
\end{center}
\caption{Фотография установки: вид сверху}\label{device2:pict}
\end{figure}

В обеих крышках и в боковине просверлены сквозные технологические отверстия диаметром 6~мм, оснащенные с внешней стороны приваренными резьбовыми штуцерами. В верхней крышке находится 29 отверстий, в нижней~---13, в боковине~---6. Эти отверстия могут использоваться как для монтажа различных датчиков, так и для обеспечения отбора или закачки флюида в коллектор. Поровое давление в модели измеряется через технологические отверстия, расположенные в нижней крышке установки, с помощью тензопреобразователей. Технологические отверстия заполняются водой и перед заливкой гипса закрываются поролоновыми вкладышами. Вкладыши на 5~мм выступают над поверхностью нижнего основания и после заливки гипса оказываются вмонтированы в него, обеспечивая передачу порового давления к тензопреобразователям. Схема расположения датчиков показана на рисунке~\ref{device3:pict}.

\begin{figure}[hb]
\begin{center}
\epsfig{file=figs/picts/5, width=.5\textwidth}
\end{center}
\caption{Схема расположения датчиков в нижней и верхней крышках}\label{device3:pict}
\end{figure}

Вспомогательные скважины, необходимые для прокачки насыщенного раствора сульфата кальция (гипса) через образец и для создания поля порового давления в нем, формируются при отливке образца путем помещения в него заранее вставок из фторопласта диаметром 15~мм в центрально симметричные технологические отверстия (рисунок~\ref{device2:pict}). После затвердевания гипса вставки вынимаются, а сами образовавшиеся скважины закрываются фторопластовыми крышками.

Центральная скважина представляет собой латунную трубку диаметром 16~мм, которая герметично вставляется в нижнюю крышку установки. Трубка имеет возможность свободно вращаться вокруг вертикальной оси, позволяя ориентировать затравку трещины гидроразрыва пласта (ГРП) в заданном направлении. Верхний торец трубки закрыт винтовой пробкой. В средней части трубки проделана вертикальная прорезь, в которую вставляется сложенная вдвое тонкая латунная сетка, служащая затравкой трещины ГРП. Размер лепестков сетки составляет 8х8~мм. Углы лепестков срезаны примерно на 2~мм. После заливки гипса мы получаем обсаженную скважину с перфорированной стенкой и затравкой трещины ГРП.

Эксперименты, как правило, проводятся на третьи сутки после заливки гипса, когда высохнет гипс. После сборки экспериментальной установки модель нагружается небольшим вертикальным давлением (1~МПа), затем задается необходимое давление в боковых камерах. После этого вертикальное давление поднимается до рабочего значения. Перед началом эксперимента проводится дополнительное насыщение гипсового образца жидкостью под постоянным давлением закачки около 1~МПа на технологической нагнетательной скважине. Критерием завершения процесса насыщения служит стабилизация расхода на добывающей скважине и давлений в точках измерения порового давления. Обычно насыщение продолжается около 1~часа. Непосредственно после завершения процесса насыщения проводится эксперимент по ГРП.

\subsection{Модельная среда}

Выбор среды, моделирующей коллектор, определяется целью и постановкой решаемых экспериментальных задач. Целью данного исследования является экспериментальное моделирование ГРП с возможностью переноса результатов экспериментов на пластовые условия. Выбор материала модельного образца связан с двумя основными условиями:
\begin{itemize}
\item
критерии подобия, отвечающие за возможность переноса данных с эксперимента на пласт~\cite{cleary1994};
\item
технологические факторы, связанные с возможностью изготовления экспериментальных образцов.
\end{itemize}
С этой точки зрения смесь на основе гипса с добавкой портландцемента является хорошим выбором. Хорошая текучесть смеси и отсутствие усадки при затвердевании позволяет добиться плотного контакта со стенками установки. Для замедления "схватывания" гипса в воду для приготовления смеси добавляется лимонная кислота в концентрации 2~г/дм$^3$.

В рассматриваемых далее экспериментах среда обладает следующими свойствами.
\begin{description}
%
\item $\nu_\text{dyn} \;\;\,  = 0.25$~-- динамический коэффициент Пуассона,
\item $\nu_\text{st} \;= 0.2$~-- статический коэффициент Пуассона,
\item $E_\text{dyn} \,\,    = 7.5 \times 10^{9}\;$  [Па]~-- динамический модуль Юнга, 
\item $E_\text{st} \,\,    = 3.7 \times 10^{9}\;$  [Па]~-- статический модуль Юнга, 
\item $k \;\;\,    = 2.4 \;\,$  [мД]~-- коэффициент абсолютной проницаемости пласта.

%
\end{description}

Описание экспериментов по определению данных прочностных и фильтрационных характеристик образца можно найти  в~\cite{trimonova2017, trimonova2018}. 

\subsection{Описание эксперимента}

В настоящем разделе приведено описание одного из экспериментов. Эксперимент проводился без гидроразрыва пласта для  проверки теории однофазной фильтрации на кривых падения давления. Для этого во вспомогательные скважины сначала закачивался растрор гипса с целью насыщения образца и создания в нем стационарного поля порового давления, после чего давление в нагнетательной скважине сбрасывалось. Соответственно поровое давление в датчиках начинало спадать, и это фиксировалось в течение всего процесса падения давления. Давления в датчиках записывалось каждые 0.01 секунды. 

Для создания этого эксперимента перед заливкой гипса на дно установки фиксировались вспомогательные скважины в точках с координатами [( 0.057, 0.127 ), (-0.057, -0.127)] и центральная скважина. На дно установки монтировались датчики давления согласно рисунку 4 в точках с координатами [(0.057, -0.127), (0.07, 0.0), (-0.057, 0.127), (0.0, 0.127), (0.0, -0.185), (0.065, 0.065), (-0.121, 0.121), (0.0, 0.07), (0.121, 0.121), (0.127, 0.0), (0.0, -0.07), (0.0, 0.0), (-0.185, 0.0)].  Измеренное в датчиках давление соответствовало давлению на расстоянии 4 мм от дна образца. Далее в установку заливался гипс и высушивался в течение 2-3 дней. После затвердевания образца вспомогательные боковые скважины удалялись. Далее, как упоминалось выше, сверху на образец подавалось давление (20~атм, где 1~атм~=~101335~Па), имитируя литостатическое давление. Давление в боковых камерах не задавалось. После этого во вспомогательную скважину с координатами центра $(0.057, 0.127 )$ и радиусом 7.5~мм закачивался раствор гипса с постоянным давлением (14.5~атм). Другая скважина была соединена с атмосферой. Насыщение образца продолжалось до установления в нем стационарного режима. Установление режима определялось по давлению в  центре образца, в точке с координатами $(0, 0)$. Когда давление в ней достигало значения, равного полусумме давлений, заданных в боковых скважинах. После этого раствор в нагнетательную скважину переставал подаваться. Кривые давления в датчиках порового давления фиксировались в течение всего эксперимента (рисунок~\ref{device4:pict}) . Длительность всего эксперимента составила примерно 220~мин. 

\begin{figure}[hb]
\begin{center}
\epsfig{file=figs/picts/6, width=.6\textwidth}
\end{center}
\caption{Кривые давления в датчиках порового давления}\label{device4:pict}
\end{figure}



