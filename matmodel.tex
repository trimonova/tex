% matmodel.tex
% 

%\subsection{Основные допущения}
%\label{sec:assumptions}

%В настоящем разделе описана модель Био пороупругой среды.
%Приведенная модель является физически и геометрически линейной, то
%есть
%соответствует случаю малых отклонений полей, описывающих состоние
%среды, от своих равновесных (опорных) значений.


% Основные допущения модели имеют вид:
% % 
%  \begin{itemize}
%  \item Напряженно-деформированное состояние пласта основано на
%    физически и геометрически линейной модели. То есть рассматривается
%    случай бесконечно малых деформаций,
% %
% \(
% \|\nabla\tpr\vc{u}\| \ll 1
% \),
% %
% где $\vc{u}$~-- вектор перемещений точек среды,
% и бесконечно малых перемещений
% %
% \(
% \left\|{\vc{u}}/{L}\right\| \ll 1,
% \)
% %
% где $L$~-- характерный размер вмещающей среды.

% Это допущение позволяет отождествить лагранжевы ($\vc{X}$) и эйлеровы
% ($\vc{x} = \vc{x}(\vc{X}, t)$) координаты точек среды в
% деформированном состоянии, $\vc{x}\approx \vc{X}$, и элементарные
% объемы среды $\omega\approx \Omega$, $\omega$~-- элементарный объем в
% лагранжевых координатах, $\Omega$~-- в эйлеровых.

% Из этих допущений также следует линейная связь между тензорами
% деформаций и перемещений, см. раздел~\ref{sec:poroelast}.

% \item Задача рассматривается в неизотермическом приближении,
% при этом отклонения температуры от начальной предполагаются малыми,
% %
% \(
% \left|\Delta \Theta / {\Theta_0}\right| \ll 1
% \),
% %
% где для какой либо величины $f$ символом 
% $f_0$ обозначено ее характерное значение в начальном состоянии,
% $\Delta f = f - f_0$~-- отклонение от него.

% \item Пластовый флюид является однофазным и слабо сжимаемым, то есть
% %
% \(
% \left| \Delta \rho_f / {\rho_f^0}\right| \ll 1.
% \)
% %
% Приведенные соотношения включают, как частный случай, случай
% несжимаемого пластового флюида.

% \item Пористость (см. раздел~\ref{sec:vol}) в ходе деформации меняется
%   мало, 
% %
% \(
% \left| \Delta \phi / {\phi_0}\right| \ll 1
% \).
% %
% \end{itemize}
% % 
% %


% %%%%%%%%%%%%%%%%%%%%%%%%%%%%%%%%%%%%%%%%%%%%%%%%%%%%%%%%%%%%%%%%%%%%%%%%%%%%%%%%%%%%%%%%%%%%%%%%%%

%\subsection{Вмещающая среда}\label{sec:poroelast}
%
%
%Будем считать, что динамическими эффектами (силами инерции) при
%описании напряженно-деформированного состояния вмещающей среды можно
%пренебречь (то есть процесс является квазистационарным).
%
%Рассмотрим случай малых деформаций.
%
%Пусть $\vcu = \vcu(\vc{x},t)$~-- вектор перемещений точек вмещающей
%(трещину) среды.  Симметричный тензор малых деформаций $\tn{E}$ имеет
%вид:.%\marginpar{Эйлеров или лагранжев, инфинитизимальный?}
%%
%\begin{equation*} 
%\tn{E}(\vc{u}) = \frac{1}{2} \left[\mathbf{\nabla} \tpr \vc{u} + (\mathbf{\nabla}\tpr \vc{u} )^{T} \right].  
%\end{equation*} 
%
%
%Уравнения механического равновесия (уравнения закона сохранения
%импульса) имеют вид:
%%
%\begin{equation} 
%\label{eq:equil} % %\label{GrindEQ__1_2_4_} 
%\mathbf{\nabla} \dpr \tn{T} + \rho \vc{g} = 0, 
%\end{equation} 
%где $\rho = \rho(\vc{x})$~-- плотность вмещающей среды, $\vc{g}$~-- ускорение свободного падения,
%$\tn{T}$~-- тензор напряжений.
%%
%Описанные уравнения должны быть дополнены граничными условиями.
%
%В общем случае анизотропного пороупругого насыщенного тела определяющие соотношения 
%в дифференциальном виде имеют вид~\cite{coussy_2004}:
%%
%\begin{gather}
%d\tn{T} = \tn{C}:d\tn{E} - \tn{B}dp - \tn{C}:\tn{A}d\Theta, \label{eq:dsigma}\\
%d\phi = \tn{B}:d\tn{E} + \frac{1}{N} dp - 3\alpha_\phi d\Theta, \label{eq:dphi} \\
%dS_s  = \tn{C} : \tn{A} : d\tn{E} - 3\alpha_\phi dp + \frac{C}{\Theta_0}d\Theta\label{eq:ds},
%\end{gather}
%%
%где
%%
%$\tn{T}$~-- тензор напряжений, 
%$\tn{E}$~-- тензор деформаций,
%$\tn{C} = [C_{ijkl}]$~-- тензор упругих коэффициентов 4-го ранга (со стандартными свойствами симметрии $C_{ijkl}=C_{klij}=C_{jilk}$),
%$\tn{B} = B_{ij}$~-- симметричный тензор Био, 
%$\tn{A}=[\alpha_{ij}]$~-- симметричный тензор коэффициентов термического расширения, 
%$\Theta$~-- температура, 
%$\phi$~-- пористость,
%$\alpha_\phi$~-- объемный коэффициент термического расширения порового пространства,
%$N$~-- модуль Био, связывающий изменение пористости с изменением давления флюида, 
%$C$~--  коэффициент теплоемкости скелета при постоянном объеме, 
%$S_s$~-- энтропия скелета. 
%
%%
%Интегрирование соотношений~\eqref{eq:dsigma}-\eqref{eq:ds} от начального (опорного) состояния до текущего дает:
%% %
%% \begin{gather*}
%% \tn{T} - \tn{T}_0 = \tn{C}:\tn{E} - \tn{B}(p-p_0) - \tn{C}:\tn{A}(\Theta-\Theta_0),\\
%% \phi - \phi_0 = \tn{B}:\tn{E} + \frac{1}{N} (p-p_0) - 3\alpha_\phi(\Theta-\Theta_0), \\
%% S_s - S_s^0 = \tn{C} : \tn{A} : \tn{E} - 3\alpha_\phi (p-p_0) + \frac{C}{\Theta_0}(\Theta-\Theta_0).
%% \end{gather*}
%% %
%\begin{gather*}
%\Delta\tn{T} = \tn{C}:\tn{E} - \tn{B} \Delta p - \tn{C}:\tn{A}\Delta \Theta,\;
%\Delta \phi = \tn{B}:\tn{E} + \frac{1}{N} \Delta p - 3\alpha_\phi\Delta \Theta, \\
%\Delta S_s = \tn{C} : \tn{A} : \tn{E} - 3\alpha_\phi \Delta p + \frac{C}{\Theta_0}\Delta\Theta,
%\end{gather*}
%%
%где $\Delta f = f - f_0$ для какой-либо величины $f$.
%В случае изотропной среды имеем:
%%
%$\tn{C} = \lambda \tn{I}\tpr \tn{I} + 2G \tn{I}$,
%$\tn{B} = b\tn{I}$, $\tn{A} = \alpha \tn{I}$,
%%
%% и определяющие соотношения можно записать в виде:
%% %
%% \begin{gather*}
%% \tn{T} - \tn{T}_0 = 
%% \left[ \lambda \epsilon \tn{I} + 2\mu\tn{E} \right] - b (p-p_0) - 3\alpha K(\Theta-\Theta_0)\tn{I}\\
%% \phi - \phi_0 = b \epsilon + \frac{1}{N} (p-p_0) - 3\alpha_\phi(\Theta-\Theta_0), \\
%% S_s - S_s^0 = 3\alpha K \epsilon  - 3\alpha_\phi (p-p_0) + \frac{C}{\Theta_0}(\Theta-\Theta_0),
%% \end{gather*}
%%
%где 
%$\lambda$, $G$~-- коэффициенты Ламе скелета, 
%$\epsilon$~-- объемная деформация,
%%
%\(
%\epsilon = \tn{E}:\tn{I} = E_{ii}
%\),
%%
%\(
%\lambda = K - (2/3)G
%\),
%%
%-- объемный модуль упругости.
%Здесь и далее индексом <<$0$>> отмечены значения параметров в начальный момент времени или их опорные значения.
%
%
%Плотность $\rho$ в~\eqref{eq:equil} 
%зависит от плотности скелета и флюида и пористости вмещающей среды: 
%%
%\(
%\rho = (1-\phi) \rho_s +  \phi \rho_f,
%\)
%%
%где
%$\rho_s$~-- плотность скелета,
%$\rho_f$~-- плотность флюида (см. раздел~\ref{sec:fluid_flow}).
%
%% %%%%%%%%%%%%%%%%%%%%%%%%%%%%%%%%%%%%%%%%%%%%%%%%%%%%%%%%%%%%%%%%%%%%%%%%%%%%%%%%%%%%%%%%%%%%%%%%%%
%\subsection{Уравнения течения жидкости во вмещающей среде }\label{sec:fluid_flow}
%
%В этом разделе рассмотрены уравнения течения (фильтрации) жидкости в пороупругой насыщенной среде,
%которые представляют собой уравнения закона сохранения массы флюида.
%
%Основное их отличие от <<обычных>> уравнений фильтрации в пористой среде заключается в том, 
%что они учитывают изменение содержание жидкости (флюида) за счет деформации скелета породы.
%
%\subsubsection{Геометрические соотношения}\label{sec:vol}
%
%При деформации породы происходит изменение объема, занятого
%флюидом. Это изменение связано как с изменением объема скелета, так и
%с изменением объема порового пространства за счет деформации скелета.
%
%В рамках рассматриваемого приближения бесконечно малых деформаций
%будем считать, что эйлеров и лагранжев подход описания деформации
%среды совпадают, все величины будем считать лагранжевыми (если не
%указано обратное).
%
%
%Пусть $d\omega$~-- элементарный объем пористой насыщенной среды в исходном состоянии, а $d\Omega$~-- в деформированном.
%Пусть в деформированном состоянии флюид занимает объем 
%%
%\(
%n d\Omega,
%\)
%%
%где $n$~-- (эйлерова) пористость.  Введем лагранжеву (отнесенную к
%исходному состоянию) пористость $\phi$ соотношением:
%%
%\(%\begin{equation}\label{eq:vol1}
%\phi d\omega = n d\Omega.
%%\end{equation}
%\)
%%
%Отсюда 
%%
%\(
%%\begin{equation}\label{eq:vol2}
%\phi = J n,
%%\end{equation}
%\)
%% 
%где $J = \det \tn{F}$, 
%$\tn{F} = \tn{I} + \nabla\tpr \vc{u}$~-- тензор градиентов деформаций.
%Величину 
%%
%\(
%e = {n}/(1-n)
%\)
%%
%будем называть приведенной пористостью (void ratio).
%
%В предположении бесконечно малых деформаций имеем:
%%
%\(
%J = \det F \approx 1 +  \nabla\dpr \vc{u} =  1 + \tr\tn{E}
%\), 
%где след тензора деформаций $\varepsilon = \tr\tn{E}$
%характеризует изменение объема скелета породы, так что
%%
%\(
%d\Omega = (1 + \varepsilon) d\omega.
%\)
%%
%
%Наблюдаемое макроскопическое изменение объема скелета обусловлено как изменением порового объема, так и 
%изменением объема матрицы (хотя последняя величина не является наблюдаемой в макроскопическом эксперименте).
%Аналогично ранее полученным выражениям, для изменения объема матрицы имеем:
%%
%\(
%d\Omega^S = (1+\varepsilon^S) d\omega^S
%\).
%%
%С учетом определения эйлеровой и лагранжевой пористости имеем:
%%
%$d\Omega^S = (1-n) d\Omega = d\Omega - \phi d\omega$,
%$d\omega^S = (1-\phi_0) d\omega$,
%%
%где $\phi_0$~-- начальная пористость.
%В выражениях выше индексом <<S>> обозначены величины, отнесенные не к 
%пористой матрице, а к непосредственно \emph{скелету}, то есть (твердой
%деформируемой) среде, из которой состоит пористая <<губка>>,
%образующая матрицу.
%
%Комбинируя полученные выражения, получим выражение для описания
%баланса объема в пористой насыщенной среде:
%%
%\(
%\epsilon = (1-\phi_0)\epsilon^S + (\phi-\phi_0).
%\)
%%
%
%Если допустить, что изменение порового объема в основном обусловлено
%деформацией скелета и изменением объема матрицы можно пренебречь, то
%последнее выражение примет вид 
%%
%$\epsilon = \phi-\phi_0$.
%В терминах приведенной пористости $e$ последнее соотношение можно
%записать в виде 
%$\epsilon = (e - e_0)/(1 + e_0)$.
%
%
%\subsubsection{Определяющие соотношения для флюида}
%
%
%Уравнение состояния для флюида в форме уравнения для внутренней энергии  имеет вид:
%%
%\begin{equation*}%\label{eq:fluid_eos_base}
%de_f = -p d\left(\frac{1}{\rho_f}\right) + \Theta d s_f.
%\end{equation*}
%%
%И далее:
%%
%\begin{equation}
%\label{eq:fluid_eos}
%\frac{d\rho_f}{\rho_f} = \frac{1}{K_f} dp - 3\alpha_f d\Theta,\quad ds_f = -3\alpha_f \frac{dp}{\rho_f} + C_p\frac{d\Theta}{\Theta},
%\end{equation}
%%
%где $p$~-- давление флюида, $\rho_f$~-- его плотность, 
%$3\alpha_s$~-- коэффициент температурного расширения флюида, 
%$\Theta$~-- его температура, $s_f$~-- энтропия, $C_p$~-- теплоемкость при постоянном давлении, $K_f$~-- объемный модуль сжатия флюида.
%% %
%% \[
%% K_f = - \frac{1}{1/\rho_f} \dudx{(1/\rho_f)}{p},\quad 
%% -3\alpha_f = \frac{1}{(1/\rho_f)} \dudx{(1/\rho_f)}{\Theta}, \quad
%% C_p = \Theta\dudx{s_f}{p} = \dudx{h_f}{\Theta},
%% \]
%% %
%% $h_f$~-- энтальпия флюида:
%% %
%% \begin{gather*}
%% h_f = e_f + \frac{p}{\rho_f};\quad
%% h_f = h_f(p, s_f),\quad \frac{1}{\rho_f} = \dudx{h_f}{p},\quad \Theta = \frac{h_f}{s_f}.
%% \end{gather*}
%% % 
%
%В дальнейшем будем считать, что отклонения от начального состояния малы и  флюид слабо сжимаем, то есть
%%
%$K_f = \const$, $C_p = \const$, $\alpha_f = \const$
%-- заданные параметры.
%%
%В этом случае из~\eqref{eq:fluid_eos} имеем:
%%
%\[
%\frac{\rho_f - \rho_f^0}{\rho_f^0} = \frac{p-p_0}{K_f} - 3\alpha_f \frac{\Theta - \Theta_0}{\Theta_0},\quad
%s_f - s_f^0 = -3\alpha_f\frac{p-p_0}{\rho_f^0}  + C_p\frac{\Theta - \Theta_0}{\Theta_0}.
%\]
%%
%%где индексом <<$0$>> помечены опорные (начальные) значения параметров.
%
%
%В частном случае несжимаемой жидкости ($K_f\to\infty$, $\alpha_f = 0$) 
%и малых изменений температуры имеем:
%$\rho_f = \rho_f^0$, $\Delta s_f = C_p \Delta\Theta/\Theta_0$.
%%
%% \[
%% \rho_f = \rho_f^0,\; s_f - s_f^0 = C_p \frac{\Theta-\Theta_0}{\Theta_0},
%% \]
%% %
%
%\subsubsection{Определяющие соотношения для насыщенной  среды}
%
%Пусть $m_f$~-- масса жидкости, отнесенная к единице объема в
%лагранжевых координатах и определяемая соотношением
%$\rho_f n d\Omega = m_f d\omega$,
%%
%откуда 
%%
%\begin{equation*}
%%\label{eq:mf} 
%m_f = \rho_f \phi.
%\end{equation*}
%%
%
%Из последнего соотношения следует, что 
%%
%\[
%\frac{d m_f}{\rho_f} = d\phi + \phi \frac{d \rho_f}{\rho_f}.
%\]
%%
%Тогда из~\eqref{eq:fluid_eos} следует
%%
%\[
%d\phi = \frac{d m_f}{\rho_f} - \phi \left( \frac{d p}{K_f} + 3\alpha_f d\Theta \right),\quad 
%d(m_f s_f) = s_f dm_f - 3 \phi \alpha_f dp + m_f C_p \frac{d\Theta}{\Theta}.
%\]
%%
%
%С учетом этого соотношения пористость $\phi$ может быть исключена из уравнений~\eqref{eq:dsigma}-\eqref{eq:ds},
%которые с учетом~\eqref{eq:fluid_eos} можно будет представить в виде:
%%
%\begin{gather}
%d\tn{T} = \tn{C}:d\tn{E} - \tn{B}dp - \tn{C}:\tn{A}d\Theta, \label{eq:dsigma:1}\\
%\frac{dm_f}{\rho_f} = \tn{B}:d\tn{E} + \frac{1}{M} dp - 3\alpha_m d\Theta, \label{eq:dphi:1} \\
%dS  = s_f dm_f + \tn{C} : \tn{A} : d\tn{E} - 3\alpha_m dp + \frac{C_d}{\Theta_0}d\Theta\label{eq:ds:1},
%\end{gather}
%%
%где 
%%
%\[
%S = S_s + m_f s_f,\; \frac{1}{M} = \frac{1}{N} + \frac{\phi}{K_f},\;\; 
%\alpha_m = \alpha_\phi + \phi \alpha_f,\;\;
%C_d = C + m_f C_p.
%\]
%
%С учетом условий малых изменений свойств насыщенной среды, из~\eqref{eq:dphi:1} следует:
%%
%\begin{equation}
%\label{eq:vf:1}
%v_f = \tn{B}: (\tn{E} - \tn{E}_0) + \frac{1}{M} (p-p_0) - 3\alpha_m (\Theta-\Theta_0), %\label{eq:dphi:1} \\
%\end{equation}
%%
%где $v_f$~-- относительное изменение объема жидкости в элементарном объеме насыщенной среды, которое определяется как
%%
%\begin{equation}
%\label{eq:vf:2}
%v_f = \frac{m_f - m_f^0}{\rho_f^0}.
%\end{equation}
%%
%
%
%\subsubsection{Основные уравнения. Закон сохранения массы флюида}
%
%Уравнения механического равновесия (уравнения закона сохранения
%импульса) для насыщенной среды были рассмотрены в
%разделе~\ref{sec:poroelast}.  С учетом сделанных допущений плотность в
%правой части уравнения~\eqref{eq:equil} задается выражением:
%%
%$\rho = (1-\phi_0) \rho_s^0 + \phi_0 \rho_f^0$.
%%
%
%Уравнения закона сохранения массы для флюида имеют вид (с учетом
%допущений о линейности модели и при отсутствии источников):
%%
%\begin{equation*}
%%\label{eq:mb}
%\dudx{m_f}{t} + \nabla\dpr (\vc{w}_m) = 0,\quad 
%\vc{v} = \frac{\vc{w}_m}{\rho_f^0} = \tn{K}\dpr (-\nabla p + \rho_f^0 \vc{g}),
%\end{equation*}
%%
%где
%%
%\(
%\vc{w}_m = \rho_f^0 \vc{v}
%\)
%%
%-- вектор плотности потока массы,
%$\vc{v}$~-- скорость фильтрации, определяемая законом Дарси,
%%
%% \[
%% \vc{v} = \frac{\vc{w}}{\rho_f^0} = \tn{K}\dpr (-\nabla p + \rho_f^0 \vc{g}),
%% \]
%%
%$\rho_f^0$~-- плотность флюида в начальном состоянии, 
%$m_f$~-- масса флюида в единице объема насыщенной среды, 
%$\tn{K}$~-- симметричный тензор коэффициентов проницаемости.
%
%Величина $m_f$ как функция состояния и свойств среды определяется соотношениями~\eqref{eq:vf:1}, \eqref{eq:vf:2}.
%
%\subsection{Закон сохранения энергии}
%
%Считая, что в элементарном объеме насыщенной среды скелет и флюид 
%находятся в состоянии локального термодинамического равновесия,
%уравнение закона сохранения энергии (при отсутствии внешних источников энергии) примет вид:
%%
%\[
%\dudx{e}{t} + \nabla\dpr(h_f \vc{w}_m + \vc{q}_\Theta) = 0,\quad
%\vc{q}_\Theta = -\tn{\Lambda} \nabla \Theta,
%\]
%%
%где
%$h_f$~-- энтальпия флюида,
%$\vc{q}_\Theta$~-- вектор плотности теплового потока за счет эффектов теплопроводности,
%$e$~-- полная внутренняя энергия элемента объема насыщенной среды,
%%
%\(
%e = \rho_s (1-\phi)e_s + \rho_f\phi e_f
%\),
%$\Lambda$~-- симметричный тензор коэффициентов теплопроводности.
%
%Зависимости внутренней энергии от давления и температуры (уравнения
%состояния) могут быть получены непосредственно из
%уравнений~\eqref{eq:fluid_eos}, \eqref{eq:dsigma}-\eqref{eq:ds}.

%%%%%%%%%%%%%%%%%%%%%%%%%%%%%%%%%%%%%%%%%%%%%%%%%%

%\subsection{Упрощенная математическая модель}

В настоящем разделе рассмотрим несколько упрощенный вариант модели, подробно описанной в ~\cite{borisov_2017}. Именно такая модель будет использовнна ниже при
описании вычислительных алгоритмов и проведении тестовых расчетов. 

Указанные упрощения не являются принципиальными и носят, скорее,
методический характер: упрощенная модель является изотермической, 
а вмещающая среда является однородной и изотропной.

Пусть задача решается в пространственной области $\Omega$.
Уравнения, составляющие линейную модель Био, имеют вид:
%
\begin{gather}
\label{eq:elast}
\nabla \dpr\tn{T} +\rho\vc{g} = 0,\\
%
\label{eq:flow}
\dudx{}{t}\left( m_f \right) + \nabla \dpr \left( - \rho_f^0 \frac{K}{\mu}\nabla p\right) = 0,
%
\end{gather}
%
где
%
$\tn{T} = \tn{T}' - b p \tn{I}$~-- тензор полных напряжений, $p$~-- давление флюида,
%
\(
\tn{T}' = \tn{C}:\tn{E}
\)
%
~-- тензор эффективных напряжений, $\tn{C}=\const$~-- тензор упругих коэффициентов, 
$\tn{E}(\vc{u}) = \frac{1}{2} \left[\mathbf{\nabla} \tpr \vc{u} + (\mathbf{\nabla}\tpr \vc{u} )^{T} \right] $-- линейный тензор деформаций, $\vcu = \vcu(\vc{x},t)$~-- вектор перемещений.
%
$m_f$~-- массы флюида в единице объема насыщенной среды,
%
\[
m_f = \rho_f^0 v_f, \quad v_f = b\epsilon + \frac{1}{M} p,\quad \epsilon = \tn{E}:\tn{I};
\]
%
$b=\const$ и $M=\const$~-- параметры Био,
 $\epsilon$~-- объемная деформация;
%
$\rho = \const$~-- плотность насыщенной среды,
%
$\rho = \rho_s^0 (1-\phi_0) + \rho_f^0\phi_0$, где
%
$\rho_s^0=\const$~-- плотность скелета, $\rho_f^0=\const$~-- плотность флюида, $\phi_0=\const$~-- начальная пористость;
%
$K=\const$~-- коэффициент проницаемости пористой среды,
%
$\mu$~-- вязкость флюида,
%
$\vc{g}$~-- ускорение свободного падения, 
%
$\vc{g} = g\vc{e}_g,\quad g = \const$, 
$\vc{e}_g$---вектор направления силы тяжести.
%

Постановка задачи должна быть дополнена граничными условиями на границе $\partial\Omega$ области
$\Omega$, а также начальными условиями в момент времени~$t=0$. 
Для <<упругой>> части задачи (уравнение~\eqref{eq:elast}) на границе
области могут быть заданы компоненты (полных) нормальных напряжений
либо компоненты поля перемещений. Для <<фильтрационной>> части задачи
(уравнение~\eqref{eq:flow})
могут быть заданы либо значения давления, либо нормальная компонента
вектора плотности потока массы флюида.

Первичными неизвестными задачи являются поле перемещений вмещающей среды $\vc{u}$ и давление
флюида в пласте $p$.


%%%%%%
\endinput
% EOF