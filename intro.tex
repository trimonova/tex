%
% intro.tex
% 
% Введение
% 

В настоящее время математическое моделирование является
одним из основных инструментов, используемых для анализа и оптимизации 
процессов разработки нефтегазовых месторождений. Традиционно 
основное внимание при этом уделяется математическому моделированию
фильтрационных процессов, то есть моделированию течения многофазного
многокомпонентного флюида в деформируемом пористом пласте~\cite{aziz_2004}.
Учет механического состояния пласта в фильтрационных моделях 
осуществляется путем задания зависимости пористости пласта 
(то есть объемной концентрации пустот, содержащих флюид) от давления
флюида. Для решения фильтрационных задач описания такой точности
обычно достаточно.

Однако известен целый ряд ситуаций, анализ которых требует 
полноценного учета напряженно-деформированного состояния пласта.
В этом случае для описания системы <<пласт>>--<<флюид>>
обычно используется модель пороупругой среды,
которая позволяет описать фильтрацию флюида в порах совместно с
полноценной механической моделью напряженно-деформированного состояния пласта.

В современном виде такие модели были предложены в работах
М.~Био \cite{biot_1941}. 
Модель Био описывает протекающие совместно процессы деформации упругой
среды (матрицы породы) и течения флюида в ней.
Модель является
макроскопической в том смысле, что в рамках нее принимается,
что вмещающее пороупругую среду
пространство заполнено двухфазной средой, причем одна
фаза соответствует непосредственно пористой среде, а вторая~--
содержащемуся в порах флюиду. Обе фазы присутствуют в каждой точке
физического пространства, а распределение фаз в пространстве
описывается
макроскопическими величинами типа пористости (объемной концентрации
заполненных флюидом пустот в среде). Подвижную фазу далее будем
называть флюидом. Для обозначения твердой фазы будем  использовать
термин <<вмещающая среда>> или <<матрица>>.
Для обозначения твердой фазы, то есть вещества, из которого
состоит образующая матрицу твердая деформируемая <<губка>>,
будем использовать термин <<скелет>>. Отметим, что матрица является
пористой средой, в то время как скелет имеет нулевую пористость.
В дальнейшем величины, отнесенные к твердой фазе (матрице, вмещающей среде), будем обозначать
нижним индексом <<s>>, к подвижной фазе (флюиду)~--- нижним индексом <<f>>.

В общем случае модель Био состоит из двух групп уравнений:
(I) уравнения теории (термо)упругости с учетом внутренних сил,
связанных с влиянием давления флюида в порах на
напряженно-деформированное
состояние среды и (II) фильтрации флюида в порах с учетом изменения 
объема порового пространства, заполненного флюидом, за счет деформации среды.

В настоящем препринте приведено детальное описание модели Био в ее
современном виде для случая физически и геометрически линейной упругой
среды и однофазной фильтрации в порах. При этом считается, что поры полностью
заполнены флюидом. Рассмотрены уравнения модели, имеющие вид законов
сохранения массы, импульса и энергии, приведен полный набор
определяющих соотношений. Описание модели основано на~\cite{coussy_2004}. 
Далее
рассмотрены вычислительные алгоритмы на основе метода конечных
элементов для решения уравнений модели и
результаты их применения для решения ряда тестовых задач.

Описанная модель является основой разрабатываемого в ИПМ
им. М.В. Келдыша РАН программного комплекса для
математического моделирования динамики развития трещины гидроразрыва
пласта в рамках проекта РНФ~\No~15-11-00021.

Важно отметить, что назначение разрабатываемого программного комплекса~-- решение
существенно более сложной, чем задача Био, задачи, а именно~--
математическое моделирование динамики развития трещины гидроразрыва
пласта в полной трехмерной постановке. Пороупругая часть задачи
является основной составляющей, но при этом и наиболее
простой. По этой причине для решения задачи Био были использованы
простейшие конечно-элементные методы, которые позволяют получить корректные
результаты моделирования, хотя, возможно, в ущерб вычислительной
эффективности как алгоритмов, так и их программной реализации.

%%%%%%
\endinput
% EOF