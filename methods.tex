%
% methods.tex
%
В рассматриваемом приближении система уравнений пороупругости
представляет собой связанную задачу для эллиптического уравнения
(описывающего напряженно-деформированное состояние насыщенной среды) и
параболического уравнения (описывающего закон сохранения массы флюида
во вмещающей среде).
Для решения указанной системы уравнений могут применяться различные
методы, 
среди которых отметим метод конечных объемов, метод конечных элементов
и метод граничных интегральных уравнений. 
Для аппроксимаций по времени чаще всего используются простые полностью
неявные схемы (например, неявный метод Эйлера); они же будут использоваться
и в данной работе.

При этом полная система уравнений может аппроксимироваться как одно
целое, либо задача <<развязывается>>, и ее решение получается в
ходе тех или иных итераций между группами уравнений фильтрации и
упругости. Последний подход чаще всего применяется в случае, если
для решения задачи используются две различные программы для расчета
фильтрационной части (например, промышленный симулятор
фильтрации) и для решения задачи теории упругости. Построение
эффективных итерационных методов решения таких задач является
отдельной проблемой~\cite{kim2009, kim2010}.

Рассмотренный ниже алгоритм 
основан на методе конечных элементов. Детали реализации 
этого метода для задач теории пороупругости широко описаны в
литературе, см., например, монографию~\cite{lewis1998} и 
работы~\cite{noorishad1982, philips2005,
zheng2003}.

Характерным свойством системы уравнений пороупругости в
рассматриваемом приближении является то, что после аппроксимации по
времени система уравнений имеет вид задачи о седловой точке~\cite{brezzi1991}.
В этом случае для устойчивости решения задачи
как в континуальном, так и в дискретном случае необходимо выполнение
так называемых $\inf$-$\sup$ условий (условий
Ладыженской--Бабушки--Бреззи)~\cite{brezzi1991}. При нарушении
этих условий в задачах пороупругости возникают численные
неустойчивости и эффекты <<блокировки>> (<<locking>>) конечномерного
решения, особенно в несжимаемом пределе, см.~\cite{philips2009,
  preisig2011, haga2012, vermeer1981}.  Простейшим примером конечного
элемента, удовлетворяющего этим условиям, является элемент
Тэйлора--Худа, в котором для аппроксимации перемещений используются
конечные элементы второго порядка, а для аппроксимации давления~--
первого~\cite{ern2009, murad1996,
  showalter2000}. Альтернативным подходом является регуляризация
исходной задачи (см., например,~\cite{white2008,commend2004, wan2002,
  xia2009}), цель которой~-- построить задачу, для которой
$\inf$-$\sup$ условия выполняются для стандартных пар пространств
(конечные элементы одинакового, первого, порядка и для поля
перемещений, и для давления).

В случае, если для
конкретной задачи $\inf$-$\sup$ устойчивые пары пространств неизвестны,
необходимы тщательный контроль особенностей решения и
проверка выполнения численных $\inf$-$\sup$ условий~\cite{chappele1993}.



\subsection{Слабая постановка задачи}

Для построения аппроксимаций задачи в дальнейшем будем использовать метод конечных элементов.
Одним из основных элементов построения аппроксимаций является 
слабая постановка задачи, которая будет рассмотрена в данном разделе.

Сначала определим необходимые пространства, в которых будем искать
решение.  Пусть поле перемещений $\vc{u}$ принадлежит
пространству гладких векторных полей в области $\Omega$, $\vc{u}\in
V_{\vc{u}} = V_{\vc{u}}(\Omega)$, поле давлений $p$ во вмещающей среде
также принадлежит пространству гладких в $\Omega$ функций,
$p \in V_{p} = V_{p}(\Omega)$.
%
Все указанные пространства могут быть точно охарактеризованы в терминах пространств Соболева нужной гладкости.
Однако мы не будем этого делать в силу того, что исследование теоретических вопросов, связанных с анализом 
существования и единственности решения и скоростью сходимостью численных аппроксимаций, в данной работе не рассматриваются.

Перейдем к построению слабой постановки задачи. Для простоты будем считать,
что для всех переменных, определенных в $\Omega$, заданы
однородные главные граничные условия (то есть перемещения и поле
давления).
Учет естественных граничных условий не представляет труда
и может быть выполнен стандартными способами.
 
Слабая форма уравнения~\eqref{eq:elast} имеет вид:
%
\begin{gather*}
\int\limits_\Omega (\tn{C}:\tn{E}(\vc{u}) - b
p\tn{I}):\tn{E}(\delta\vc{u})\, d\Omega  = 
\int\limits_\Omega\rho\vc{g}\delta\vc{u}\, d\Omega,
\quad \delta\vc{u}\in V_{\vc{u}}.
\end{gather*}

Аналогично,  слабая форма закона сохранения массы флюида во вмещающей среде будет иметь вид:
%
\begin{equation}\label{eq:poro:1}
\int\limits_{\Omega} \dudx{}{t}(m_f) \delta p \,d\Omega + 
\int\limits_\Omega \rho_f^0\frac{K}{\mu} \nabla p \dpr \nabla \delta p\, d\Omega = 0,\quad \delta p\in V_p.
\end{equation}
%

Приведенная система уравнений содержит 2 уравнения относительно 2 неизвестных полей
$\vc{u}$ и~$p$.


Введем следующие билинейные формы:
%
\begin{equation}
\label{eq:weak:forms}
\begin{gathered}
\mathbb{A}_u(\vc{u},\delta\vc{u}) = \int\limits_\Omega \tn{E}(\vc{u}):\tn{C}:\tn{E}(\delta\vc{u}) \,d\Omega, \quad
\mathbb{A}_p(p,\vc{u}) = \int\limits_\Omega (-bp\tn{I}):\tn{E}(\delta\vc{u})\, d\Omega,\\
%
\mathbb{B}(p, \delta p) = \int\limits_\Omega \rho_f^0\frac{K}{\mu} \nabla p \dpr \nabla \delta p\, d\Omega,\quad
\mathbb{M}_u(\vc{u},\delta p) = \int\limits_\Omega (-\rho_f^0 b\delta p  \tn{I} ):\tn{E}(\delta\vc{u})\, d\Omega,\\
%
\mathbb{M}_p(p,\delta p) = \int\limits_\Omega \rho^0_f \frac{1}{M} p\delta p \, d\Omega,
%
\end{gathered}
\end{equation}
%
где $\vc{u}, \delta\vc{u}\in V_{\vc{u}}$, 
$p,\delta p\in V_{p}$.

% Введем следующие билинейные формы, определенные на соответствующих
% парах пространств:
% %
% \begin{gather}
% \label{eq:weak:forms}
% \begin{aligned}
% &\vc{u}\in V_{\vc{u}}, \delta\vc{u}\in V_{\vc{u}}:& \quad 
% & \mathbb{A}_u(\vc{u},\delta\vc{u}) = \int\limits_\Omega \tn{E}(\vc{u}):\tn{C}:\tn{E}(\delta\vc{u}) \,d\Omega, \\
% %
% %\label{eq:ww:2}
% &p\in V_{p}, \delta\vc{u}\in V_{\vc{u}}:& \quad 
% & \mathbb{A}_p(p,\vc{u}) = \int\limits_\Omega (-bp\tn{I}):\tn{E}(\delta\vc{u})\, d\Omega,\\
% %
% %\label{eq:ww:3}
% &p\in V_{p}, \delta p\in V_{p}:& \quad 
% & \mathbb{B}(p, \delta p) = \int\limits_\Omega 
% \rho_f^0\frac{K}{\mu} \nabla p \dpr \nabla \delta p\, d\Omega,\\
% %
% & \delta\vc{u}\in V_{\vc{u}}, \delta p\in V_{p},:& \quad 
% & \mathbb{M}_u(\vc{u},\delta p) = \int\limits_\Omega (-\rho_f^0 b\delta p  \tn{I} ):\tn{E}(\delta\vc{u})\, d\Omega,\\
% %
% & p\in V_{p}, \delta p\in V_{p}:& \quad 
% & \mathbb{M}_p(p,\delta p) = \int\limits_\Omega \rho^0_f \frac{1}{M}
% p \delta p \, d\Omega.
% %
% \end{aligned}
% \end{gather}
% %

Тогда слабая постановка задачи примет вид:
определить $\vc{u}\in V_{\vc{u}}$, $p\in V_p$, удовлетворяющие системе уравнений
%
\begin{equation}\label{eq:weak:full}
\begin{gathered}
     \mathbb{A}_u(\vc{u}, \delta{\vc{u}})   +  \mathbb{A}_p(p, \delta{\vc{u}})  = \vc{f}_u(\delta\vc{u}),\\
%
     \dudx{}{t}\left[\mathbb{M}_u(\vc{u},\delta p) + \mathbb{M}_p(p,\delta p)\right] +  \mathbb{B}(p,\delta p) = \vc{f}_p(\delta p),
%
\end{gathered}
\end{equation}
%
для всех допустимых $\delta\vc{u}\in V_{\vc{u}}$, $\delta p\in V_p$.
Заметим, что в последней системе уравнений в случае постоянной плотности флюида билинейные 
формы $\mathbb{M}_u$ и $\rho_f^0\mathbb{A}_p$ являются сопряженными
в силу того, что  сопряженными являются 
операторы дивергенции и градиента.

В приведенных выше соотношениях $\vc{f}_u(\delta\vc{u})$ и
$\vc{f}_p(\delta p)$~--- линейные функционалы, порождаемые правыми
частями уравнений~\eqref{eq:elast}, \eqref{eq:flow}. В рассматриваемом
частном случае имеем $\vc{f}_p(\delta p) = 0$,
%
\[
\vc{f}_u(\delta\vc{u}) = 
\int\limits_\Omega\rho\vc{g}\delta\vc{u}\, d\Omega.
\]




% %%%%%%%%%%%%%%%%%%%%%%%%%%%%%%%%%%%%%%%%%%%%%%%%%%%%%%%%%%%%%%%%%%%%%%%%%%%%%%%%%%%%%%%%%%%%%%%%%%
\subsection{Конечномерные аппроксимации}

Для построения конечномерных аппроксимаций задачи~\eqref{eq:weak:full}
необходимо ввести конечномерные аппроксимации пространств
$V_{\vc{u}}$, $V_p$, которые, соответственно,
будем обозначать верхним индексом <<$h$>>,
%
$V^h_{\vc{u}}\subset V_{\vc{u}}$, $V_p^h\subset V_p$.
%
Как только аппроксимации пространств выбраны, построение конечномерных аппроксимаций
не составляет труда: разложение решения и пробных функций по выбранной системе базисных
функций необходимо подставить в вариационную постановку
задачи~\eqref{eq:weak:full}. В результате получается конечномерная
система линейных обыкновенных дифференциальных уравнений относительно коэффициентов
разложения решения по выбранной системе базисных функций. Далее
указанная система уравнений аппроксимируется по времени. В результате
получается
система линейных алгебраических уравнений для определения решения
конечномерной задачи на каждом временном шаге. Указанная процедура
является стандартной и детально описана в обширной литературе по
методу конечных элементов.

Существует множество способов построения конечномерных пространств
$V_\vc{u}^h$, $V_p^h$. Традиционно в методе конечных элементов для
построения конечномерных пространств используется заданная в
аппроксимации $\Omega_h$ расчетной 
области $\Omega$ сетка конечных элементов $\omega_i$. Будем считать,
что разбиение $\Omega_h$ области $\Omega$ на конечных элементы
правильное, то есть два конечных элемента либо не пересекаются, либо
имеют общую вершину (узел), либо общее ребро, либо общую грань.

Конечные элементы обычно имеют простую форму и 
являются тетраэдрами, шестигранниками или призмами и т.д.
В разрабатываемом программном комплексе они имеют форму тетраэдров.
Для аппроксимации компонент поля
перемещений и давления используются одинаковые кусочно-линейные
базисные функции. Соответствующая пара конечномерных пространств,
вообще говоря, не является $\inf$-$\sup$ устойчивой. По этой причине
корректность расчетов, в том числе приведенных в настоящей работе,
дополнительно контролировалась с точки зрения наличия в решении
нефизических осцилляций.

После аппроксимации задачи по пространству соответствующая
система обыкновенных дифференциальных уравнений имеет вид:
%
\[
\begin{bmatrix}
0 & 0 \\
\tn{M}_u & \tn{M}_p
\end{bmatrix} \cdot
%
\begin{bmatrix}
\dot{\vc{u}}_h \\
\dot{\vc{p}}_h
\end{bmatrix} + 
%
\begin{bmatrix}
\tn{A}_u & \tn{A}_p \\
\tn{0} & \tn{B}
\end{bmatrix} \cdot
%
\begin{bmatrix}
\vc{u}_h \\
\vc{p}_h
\end{bmatrix} = \vc{F},
%
\]
%
где точкой обозначена производная по времени, $\vc{u}_h$,
$\vc{p}_h$~--
зависящие от времени векторы узловых значений конечно-элементных аппроксимаций поля
перемещений и давления соответственно;
матрицы $\tn{A}_u$, $\tn{A}_p$, $\tn{M}_u$, $\tn{M}_p$, $\vc{B}$ и~$\vc{F}$ ~-- конечномерные аппроксимации 
соответствующих билинейных форм и правой части из~\eqref{eq:weak:full}.

Пусть далее $\vc{u}_h$, $\vc{p}_h$~-- значения векторов неизвестных в момент времени
$t$, $\Hat{\vc{u}}_h$, $\Hat{\vc{p}}_h$~-- в момент времени $t+\Delta t$, $\Delta t$~-- шаг времени.
Аппроксимируя последнюю систему уравнений по времени неявным образом, получим
%
\[
\displaystyle
\begin{bmatrix}
0 & 0 \\
\tn{M}_u & \tn{M}_p
\end{bmatrix} \cdot
%
\begin{bmatrix}
\displaystyle
\frac{ \Hat{\vc{u}}_h - \vc{u}_h }{\Delta t} \\
\displaystyle
\frac{ \Hat{\vc{p}}_h - \vc{p}_h }{\Delta t}
\end{bmatrix} + 
%
\begin{bmatrix}
\tn{A}_u & \tn{A}_p \\
\tn{0} & \tn{B}
\end{bmatrix} \cdot
%
\begin{bmatrix}
\Hat{\vc{u}}_h \\
\Hat{\vc{p}}_h
\end{bmatrix} = \vc{F},
%
\]
%
или
%
%
\begin{equation}
\label{eq:slae}
\displaystyle
\begin{bmatrix}
\tn{A}_u & \tn{A}_p \\
\displaystyle \frac{1}{\Delta t}\tn{M}_u & \displaystyle \frac{1}{\Delta t }\tn{M}_p + \tn{B}
\end{bmatrix} \cdot
%
\begin{bmatrix}
\Hat{\vc{u}}_h\\
\Hat{\vc{p}}_h
\end{bmatrix} = 
\Tilde{\vc{F}}; \;
%
\Tilde{\vc{F}} = 
\vc{F} - 
%
\begin{bmatrix}
\tn{0} & \tn{0} \\
\displaystyle \frac{1}{\Delta t}\tn{M}_u & \displaystyle \frac{1}{\Delta t }\tn{M}_p + \tn{B}
\end{bmatrix} \cdot
%
\begin{bmatrix}
\vc{u}_h \\
\vc{p}_h
\end{bmatrix}.
%
\end{equation}
%
Решение этой системы линейных алгебраических уравнений относительно $\Hat{\vc{u}}_h$ и
$\Hat{\vc{p}}_h$ позволяет получить решение задачи в момент времени $t+\Delta t$.

Отметим, что существует целый ряд подходов для решения указанной системы уравнений, среди которых отметим следующие:
\begin{itemize}
\item 
Непосредственно решение системы относительно <<полного>> вектора неизвестных.
Такой подход является наиболее надежным (с точки зрения устойчивости расчета), 
однако не всегда удобен на практике, особенно если речь идет
о решения задачи, в которой уравнения Био составляют лишь часть полной системы уравнений~--
либо в случае, когда расчет упругой и фильтрационной части задачи осуществляется различными
солверами, которые не могут быть объединены в одну программу.

\item 
Организация тех или иных итераций между переменными $\Hat{\vc{u}}_h$ и  $\Hat{\vc{p}}_h$
и соответствующими группами уравнений (блочными строками системы~\eqref{eq:slae}). 
Эти итерации соответствуют 
методу простой итерации решения системы~\eqref{eq:slae} с использованием
того или иного варианта предобуславливания с помощью метода Гаусса--Зейделя.

\item 
Наконец, простейший вариант состоит в использовании ровно одной итерации Гаусса--Зейделя
на каждом шаге по времени. Этот метод обладает минимальным запасом устойчивости 
в смысле величины шага по времени.

\end{itemize}

Приведенные ниже результаты расчетов соответствуют последнему, простейшему, случаю,
в котором сначала на каждом временном шаге решается уравнение теории упругости, после чего по
обновленным данным производится расчет порового давления (т.н. дренированное расщепление,
\flqq{}drained split\frqq{}). 
Причина этого заключается в том, что, как уже отмечалось во введении, описанные в работе
математическая модель и алгоритмы 
являются частью существенно более сложного алгоритма, включающего дополнительные, по сравнению
с рассмотренной здесь задачей Био, группы уравнений.
Следует также заметить, что используемый тип расщепления является условно устойчивым~\cite{kim2010}
в зависимости от физических параметров поровой среды и флюида:
%
\begin{equation}
\label{eq:stabconddrained}
%
\cfrac{b^2 M}{\lambda+2G} \le 1.
%
\end{equation} 


\endinput
% EOF